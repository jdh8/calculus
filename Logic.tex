% Copyright (C) 2015 Chen-Pang He <http://jdh8.org/>
%
% This file may be distributed and/or modified under
%
% 1. LaTeX Project Public License
% 2. GNU Public License
%
% See the files COPYING.* for more details.

\documentclass{Slideshow}
\usepackage{metamath}

\begin{document}
\title[邏輯]{邏輯}
\maketitle

\begin{frame}{推理}
    定理需要其他已證實的定理證明?

    \[ \φ \to \ψ \to \χ \to \θ \]

    如果我們什麼都不相信,根本推理不出任何東西。

    \begin{description}
        \item[公理] 不證自明,預先設定為真。
        \item[定理] 由公理或其他已證明的定理證明為真。
    \end{description}
\end{frame}

\begin{frame}{形式系統}
    \begin{description}
        \item[文法] 用來決定符號怎麼組合成句子 (wff)\footnote{well-formed
            formula, 一般譯作「完構式」等。}
        \item[公理] 不證自明,預先設定為真的句子
        \item[推論規則] 有前提的公理
        \item[定義] 引入新符號的公理,不能使原系統證明更多句子
        \item[定理] 由公理或其他已證明的定理證明為真
    \end{description}
\end{frame}

\begin{frame}{Metamath}
    \mmhref{mmtheorems}{Metamath} 是描述形式系統的語言。文法只有簡單的取代。

    \begin{description}
        \item[句子] 用第一個字說明句子的性質
        \item[文法] 以某性質的某變數為前提,把變數組成句子
    \end{description}
\end{frame}

\section{命題邏輯}
\begin{frame}{命題邏輯}
    命題邏輯研究句子之間的關係。

    \begin{description}
        \item[蘊涵] $\left( \φ \to \ψ \right)$ 是推理的核心
        \item[否定] $\neg\φ$
        \item[等價] $\left( \φ \leftrightarrow \ψ \right)$ 用來定義新符號
        \item[且] $\left( \φ \wedge \ψ \right)$
        \item[或] $\left( \φ \vee \ψ \right)$
    \end{description}
\end{frame}

\begin{frame}{\href
    {https://zh.wikipedia.org/wiki/\%E6\%89\%AC\%C2\%B7\%E6\%AD\%A6\%E5\%8D\%A1\%E8\%B0\%A2\%E7\%BB\%B4\%E5\%A5\%87}
    {Łukasiewicz} 的公理系統
}
    \[ \tag{\mmlink{ax-1}} \left( \φ \to \left( \ψ \to \φ \right) \right) \]
    \[ \tag{\mmlink{ax-2}} \left(
        \left( \φ \to \left( \ψ \to \χ \right) \right) \to
        \left( \left( \φ \to \ψ \right) \to \left( \φ \to \χ \right) \right)
    \right) \]
    \[ \tag{\mmlink{ax-3}} \left( \left( \neg\φ \to \neg\ψ \right) \to \left( \ψ \to \φ \right) \right) \]
    \[ \tag{\mmlink{ax-mp}} \φ \ \mbox{和}\ \left( \φ \to \ψ \right) \ \mbox{能證明}\ \ψ \]
\end{frame}

\begin{frame}{定義其他邏輯符號}
    $\left( \φ \vee \ψ \right)$ 定義為 $\left( \neg\φ \to \ψ \right)$

    $\left( \φ \wedge \ψ \right)$ 定義為 $\neg\left( \φ \to \neg\ψ \right)$

    $\left( \φ \leftrightarrow \ψ \right)$ 定義為
    $\left( \left( \φ \to \ψ \right) \wedge \left( \ψ \to \φ \right) \right)$
\end{frame}

\subsection{蘊涵}
\begin{frame}{邏輯蘊涵}
    \begin{syntax}[\mmtarget{wi}]
        假說:
        \begin{align*}
            \wff \φ \\
            \wff \ψ
        \end{align*}

        斷言:
        \[ \wff \left( \φ \to \ψ \right) \]
    \end{syntax}
\end{frame}

\begin{frame}{公理 1:蘊涵的簡化}
    \begin{axiom}[\mmtarget{ax-1}]
        斷言:
        \[ \left( \φ \to \left( \ψ \to \φ \right) \right) \]
    \end{axiom}

    如果 $\φ$ 是真理,那麼加上任意前提 $\ψ$ 都成立。
\end{frame}

\begin{frame}{公理 2:蘊涵的分配律}
    \begin{axiom}[\mmtarget{ax-2}]
        斷言:
        \[ \left(
            \left( \φ \to \left( \ψ \to \χ \right) \right) \to
            \left( \left( \φ \to \ψ \right) \to \left( \φ \to \χ \right) \right)
        \right) \]
    \end{axiom}

    \href
        {https://zh.wikipedia.org/wiki/\%E6\%88\%88\%E7\%89\%B9\%E6\%B4\%9B\%E5\%B8\%83\%C2\%B7\%E5\%BC\%97\%E9\%9B\%B7\%E6\%A0\%BC}
        {Frege}
    是一位想把所有的數學運算奠基在邏輯上的數學家。這條公理是他在他的公理系統中
    首先提出的。
\end{frame}

\begin{frame}{推論規則:肯定前件 (modus ponens)}
    \begin{axiom}[\mmtarget{ax-mp}]
        假說:
        \[ \tag{min} \φ \]
        \[ \tag{maj} \left( \φ \to \ψ \right) \]

        斷言:
        \[ \ψ \]
    \end{axiom}

    這可以用來把定理 $\left( \φ \to \ψ \right)$ 的 $\ψ$ 兌現出來。
\end{frame}

\begin{frame}{定理、演繹、推論}
    要描述「若 $\sigma$ 則 $\tau$」有三種表示方法:
    \begin{description}
        \item[定理] 能證明 $\left( \sigma \to \tau \right)$
        \item[推論] $\sigma$ 能證明 $\tau$
        \item[演繹] $\left( \φ \to \sigma \right)$ 能證明 $\left( \φ \to \tau \right)$
    \end{description}

    推論是最弱的形式,需要複雜的\mmhref{mmdeduction}{演繹定理}才能化為另外兩者。
    \begin{description}
        \item[定理 $\implies$ 推論] 肯定前件 (\mmlink{ax-mp})
        \item[定理 $\implies$ 演繹] 三段論 (\mmlink{syl})
        \item[演繹 $\implies$ 定理] 同一律 (\mmlink{id})
    \end{description}
\end{frame}

\begin{frame}{公理 1 的推論形式}
    \begin{theorem}[\mmtarget{a1i}]
        假說:
        \[ \tag{a1i.1} \φ \]

        斷言:
        \[ \left( \ψ \to \φ \right) \]

        \begin{mmproof}
            \begin{mmtable}{2em}{3em}
                \statement{a1i.1}
                    $\φ$
                    \label{a1i.1}
                \statement{\mmlink{ax-1}}
                    $\left( \φ \to \left( \ψ \to \φ \right) \right)$
                    \label{a1i:ax-1}
                \statement[\ref{a1i.1}, \ref{a1i:ax-1}]{\mmlink{ax-mp}}
                    $\left( \ψ \to \φ \right)$
            \end{mmtable}
        \end{mmproof}
    \end{theorem}
\end{frame}

\begin{frame}{公理 2 的推論形式}
    \begin{theorem}[\mmtarget{a2i}]
        假說:
        \[ \tag{a2i.1} \left( \φ \to \left( \ψ \to \χ \right) \right) \]

        斷言:
        \[ \left( \left( \φ \to \ψ \right) \to \left( \φ \to \χ \right) \right) \]

        \begin{mmproof}
            \begin{mmtable}{2em}{3em}
                \statement{a2i.1}
                    $\left( \φ \to \left( \ψ \to \χ \right) \right)$
                    \label{a2i.1}
                \statement{\mmlink{ax-2}}
                    $\left(
                        \left( \φ \to \left( \ψ \to \χ \right) \right) \to
                        \left( \left( \φ \to \ψ \right) \to \left( \φ \to \χ \right) \right)
                    \right)$
                    \label{a2i:ax-2}
                \statement[\ref{a2i.1}, \ref{a2i:ax-2}]{\mmlink{ax-mp}}
                    $\left( \left( \φ \to \ψ \right) \to \left( \φ \to \χ \right) \right)$
            \end{mmtable}
        \end{mmproof}
    \end{theorem}
\end{frame}

\begin{frame}{肯定前件的演繹}
    \begin{theorem}[\mmtarget{mpd}]
        假說:
        \[ \tag{mpd.1} \left( \φ \to \ψ \right) \]
        \[ \tag{mpd.2} \left( \φ \to \left( \ψ \to \χ \right) \right) \]

        斷言:
        \[ \left( \φ \to \χ \right) \]
    \end{theorem}

    這個演繹必須額外證明,因為它要用來證明三段論。
\end{frame}

\begin{frame}{肯定前件的演繹的證明}
    \begin{mmproof}
        \begin{mmtable}{2em}{3em}
            \statement{mpd.1}
                $\left( \φ \to \ψ \right)$
                \label{mpd.1}
            \statement{mpd.2}
                $\left( \φ \to \left( \ψ \to \χ \right) \right)$
                \label{mpd.2}
            \statement[\ref{mpd.2}]{\mmlink{a2i}}
                $\left( \left( \φ \to \ψ \right) \to \left( \φ \to \χ \right) \right)$
                \label{mpd:a2i}
            \statement[\ref{mpd.1}, \ref{mpd:a2i}]{\mmlink{ax-mp}}
                $\left( \φ \to \χ \right)$
        \end{mmtable}
    \end{mmproof}
\end{frame}

\begin{frame}{三段論}
    \begin{theorem}[\mmtarget{syl}]
        假說:
        \[ \tag{syl.1} \left( \φ \to \ψ \right) \]
        \[ \tag{syl.2} \left( \ψ \to \χ \right) \]

        斷言:
        \[ \left( \φ \to \χ \right) \]
    \end{theorem}
\end{frame}

\begin{frame}{三段論的證明}
    \begin{mmproof}
        \begin{mmtable}{2em}{3em}
            \statement{syl.1}
                $\left( \φ \to \ψ \right)$
                \label{syl.1}
            \statement{syl.2}
                $\left( \ψ \to \χ \right)$
                \label{syl.2}
            \statement[\ref{syl.2}]{\mmlink{a1i}}
                $\left( \φ \to \left( \ψ \to \χ \right) \right)$
                \label{syl:a1i}
            \statement[\ref{syl.1}, \ref{syl:a1i}]{\mmlink{mpd}}
                $\left( \φ \to \χ \right)$
        \end{mmtable}
    \end{mmproof}
\end{frame}

\begin{frame}{同一律}
    \begin{theorem}[\mmtarget{id}]
        斷言:
        \[ \left( \φ \to \φ \right) \]

        \begin{mmproof}
            \begin{mmtable}{2em}{3em}
                \statement{\mmlink{ax-1}}
                    $\left( \φ \to \left( \φ \to \φ \right) \right)$
                    \label{id:ax-1:1}
                \statement{\mmlink{ax-1}}
                    $\left( \φ \to \left( \left( \φ \to \φ \right) \to \φ \right) \right)$
                    \label{id:ax-1:2}
                \statement[\ref{id:ax-1:1}, \ref{id:ax-1:2}]{\mmlink{mpd}}
                    $\left( \φ \to \φ \right)$
            \end{mmtable}
        \end{mmproof}
    \end{theorem}
\end{frame}

\begin{frame}
    \setlength{\parskip}{1ex}

    之後的定理,尤其是定理、演繹、推論的互化,我們不再一一證明。
    
    此處不證明的定理,會連結到 Metamath 官網。
\end{frame}

\begin{frame}{交換前提}
    \begin{theorem}[\mmtarget{com12}]
        假說:
        \[ \left( \φ \to \left( \ψ \to \χ \right) \right) \]

        斷言:
        \[ \left( \ψ \to \left( \φ \to \χ \right) \right) \]
    \end{theorem}
\end{frame}

\begin{frame}{公理 2 的反命題}
    \begin{theorem}[\mmtarget{pm2.86}]
        斷言:
        \[ \left(
            \left( \left( \φ \to \ψ \right) \to \left( \φ \to \χ \right) \right) \to
            \left( \φ \to \left( \ψ \to \χ \right) \right)
        \right) \]
    \end{theorem}
\end{frame}

\subsection{否定}
\begin{frame}{否定}
    \begin{syntax}
        假說:
        \[ \wff \φ \]

        斷言:
        \[ \wff \neg\φ \]
    \end{syntax}
\end{frame}

\begin{frame}{公理 3:換位律}
    \begin{axiom}[\mmtarget{ax-3}]
        斷言:
        \[ \left( \left( \neg\φ \to \neg\ψ \right) \to \left( \ψ \to \φ \right) \right) \]
    \end{axiom}

    \begin{itemize}
        \item 為什麼不規定 $\left( \left( \φ \to \ψ \right) \to \left( \neg\ψ \to \neg\φ \right) \right)$?
        \item 公理 3 有沒有改變 $\to$ 的語義?
    \end{itemize}
\end{frame}

\begin{frame}{矛盾蘊涵任意事情,演繹形式}
    \begin{theorem}[\mmtarget{pm2.21d}]
        假說:
        \[ \tag{pm2.21d.1} \left( \φ \to \neg\ψ \right) \]

        斷言:
        \[ \left( \φ \to \left( \ψ \to \χ \right) \right) \]

        \begin{mmproof}
            \begin{mmtable}{1em}{5em}
                \statement{pm2.21d.1}
                    $\left( \φ \to \neg\ψ \right)$
                    \label{pm2.21d.1}
                \statement[\ref{pm2.21d.1}]{\mmurl{a1d}}
                    $\left( \φ \to \left( \neg\χ \to \neg\ψ \right) \right)$
                    \label{pm2.21d:a1d}
                \statement[\ref{pm2.21d:a1d}]{\mmurl{con4d}}
                    $\left( \φ \to \left( \ψ \to \χ \right) \right)$
            \end{mmtable}
        \end{mmproof}
    \end{theorem}
\end{frame}

\begin{frame}{反證法}
    \begin{theorem}[\mmtarget{pm2.18}]
        斷言:
        \[ \left( \left( \neg\φ \to \φ \right) \to \φ \right) \]

        \begin{mmproof}
            \begin{mmtable}{1em}{5em}
                \statement{\mmurl{pm2.21}}
                    $\left( \neg\φ \to \left( \φ \to \neg \left( \neg\φ \to \φ \right) \right) \right)$
                    \label{pm2.18:pm2.21}
                \statement[\ref{pm2.18:pm2.21}]{\mmlink{a2i}}
                    $\left(
                        \left( \neg\φ \to \φ \right) \to
                        \left( \neg\φ \to \neg \left( \neg\φ \to \φ \right) \right)
                    \right)$
                    \label{pm2.18:a2i}
                \statement[\ref{pm2.18:a2i}]{\mmurl{con4d}}
                    $\left(
                        \left( \neg\φ \to \φ \right) \to
                        \left( \left( \neg\φ \to \φ \right) \to \φ \right)
                    \right)$
                    \label{pm2.18:con4d}
                \statement[\ref{pm2.18:con4d}]{\mmurl{pm2.43i}}
                    $\left( \left( \neg\φ \to \φ \right) \to \φ \right)$
            \end{mmtable}
        \end{mmproof}
    \end{theorem}
\end{frame}

\begin{frame}{雙重否定除去}
    \begin{theorem}[\mmtarget{notnot2}]
        斷言:
        \[ \left( \neg\neg\φ \to \φ \right) \]

        \begin{mmproof}
            \begin{mmtable}{1em}{5em}
                \statement{\mmurl{pm2.21}}
                    $\left( \neg\neg\φ \to \left( \neg\φ \to \φ \right) \right)$
                    \label{notnot2:pm2.21}
                \statement[\ref{notnot2:pm2.21}]{\mmurl{pm2.18d}}
                    $\left( \neg\neg\φ \to \φ \right)$
            \end{mmtable}
        \end{mmproof}
    \end{theorem}
\end{frame}

\begin{frame}
    \setlength{\parskip}{1ex}

    以上我們都只是把否定 ($\neg$) 拿掉。

    我們有辦法給句子加上否定嗎?
\end{frame}

\begin{frame}{否定句在後的換位}
    \begin{theorem}[\mmtarget{con2d}]
        假說:
        \[ \tag{con2d.1} \left( \φ \to \left( \ψ \to \neg\χ \right) \right) \]

        斷言:
        \[ \left( \φ \to \left( \χ \to \neg\ψ \right) \right) \]

        \begin{mmproof}
            \begin{mmtable}{2em}{4em}
                \statement{\mmlink{notnot2}}
                    $\left( \neg\neg\ψ \to \ψ \right)$
                    \label{con2d:notnot2}
                \statement{con2d.1}
                    $\left( \φ \to \left( \ψ \to \neg\χ \right) \right)$
                    \label{con2d.1}
                \statement{\mmurl{syl5}}
                    $\left( \φ \to \left( \neg\neg\ψ \to \neg\χ \right) \right)$
                    \label{con2d:syl5}
                \statement{\mmurl{con4d}}
                    $\left( \φ \to \left( \χ \to \neg\ψ \right) \right)$
            \end{mmtable}
        \end{mmproof}
    \end{theorem}
\end{frame}

\begin{frame}{雙重否定介入}
    \begin{theorem}[\mmtarget{notnot1}]
        斷言
        \[ \left( \φ \to \neg\neg\φ \right) \]

        \begin{mmproof}
            \begin{mmtable}{1em}{3em}
                \statement{\mmlink{id}}
                    $\left( \neg\φ \to \neg\φ \right)$
                    \label{notnot1:id}
                \statement[\ref{notnot1:id}]{\mmurl{con2i}}
                    $\left( \φ \to \neg\neg\φ \right)$
            \end{mmtable}
        \end{mmproof}
    \end{theorem}
\end{frame}

\begin{frame}{定義其他邏輯符號}
    $\left( \φ \vee \ψ \right)$ 應該定義為 $\left( \neg\φ \to \ψ \right)$
    \begin{description}
        \item[\mmurl{pm2.24}] $\left( \φ \to \left( \neg\φ \to \ψ \right) \right)$
        \item[\mmlink{ax-1}]  $\left( \ψ \to \left( \neg\φ \to \ψ \right) \right)$
    \end{description}

    $\left( \φ \wedge \ψ \right)$ 應該定義為 $\neg \left( \φ \to \neg\ψ \right)$
    \begin{description}
        \item[\mmurl{pm3.2im}] $\left( \φ \to \left( \ψ \to
            \neg \left( \φ \to \neg\ψ \right)
            \right) \right)$
        \item[\mmurl{impi}] $\left( \φ \to \left( \ψ \to \χ \right) \right)$
            能證明 $\left( \neg \left( \φ \to \neg\ψ \right) \to \χ \right)$
            \begin{description}
                \item[\mmlink{ax-1}] $\left( \φ \to \left( \ψ \to \φ \right) \right)$
                \item[\mmurl{idd}]   $\left( \φ \to \left( \ψ \to \ψ \right) \right)$
            \end{description}
    \end{description}
\end{frame}

\begin{frame}{Peirce 定理}
    \begin{theorem}[\mmtarget{peirce}]
        斷言
        \[ \left( \left( \left( \φ \to \ψ \right) \to \φ \right) \to \φ \right) \]
    \end{theorem}

    由此可見,公理 3 增強了 $\to$ 的語義。
\end{frame}

\subsection{同義}
\newcommand{\φtoψ}{\left( \φ \to \ψ \right)}
\newcommand{\ψtoφ}{\left( \ψ \to \φ \right)}
\newcommand{\φiffψ}{\left( \φ \leftrightarrow \ψ \right)}
\newcommand{\φdfbiψ}{\neg \left( \φtoψ \to \neg\ψtoφ \right)}

\begin{frame}{雙條件句}
    \begin{syntax}[\mmtarget{wb}]
        假說:
        \begin{align*}
            \wff \φ \\
            \wff \ψ
        \end{align*}

        斷言:
        \[ \wff \φiffψ \]
    \end{syntax}
\end{frame}

\begin{frame}{定義的藝術}
    我們希望把
    \begin{equation} \label{eq:wb}
        \φiffψ
    \end{equation}

    定義為
    \begin{equation} \label{eq:df-bi}
        \φdfbiψ
    \end{equation}

    因此我們把 \eqref{eq:wb} 代入 \eqref{eq:df-bi} 的 $\φ$,把 \eqref{eq:df-bi} 代入 \eqref{eq:df-bi} 的 $\ψ$。
\end{frame}

\begin{frame}{雙條件句的定義}
    \begin{definition}[\mmtarget{df-bi}]
        斷言:
        \begin{align*}
        &   \neg \left( \left( \φiffψ \to \φdfbiψ \right) \to \right.\\
        &   \left. \neg \left( \φdfbiψ \to \φiffψ \right) \right)
        \end{align*}
    \end{definition}

    \setlength{\parskip}{1ex}

    以後我們就可以用 $\leftrightarrow$ 來定義新符號了。

    這定義實在太長,因此我們在之後的證明中縮寫。
\end{frame}

\begin{frame}{同義蘊涵蘊涵}
    \begin{theorem}[\mmtarget{bi1}]
        斷言:
        \[ \left( \φiffψ \to \φtoψ \right) \]

        \begin{mmproof}
            \begin{mmtable}{2em}{4em}
                \statement{\mmlink{df-bi}}
                    定義 $\φiffψ$
                    \label{bi1:df-bi}
                \statement{\mmurl{simplim}}
                    $
                        \hypersetup{linkcolor=blue}
                        \left( \ref{bi1:df-bi} \to \ref{bi1:ax-mp} \right)
                    $
                    \label{bi1:simplim:1}
                \statement[\ref{bi1:df-bi}, \ref{bi1:simplim:1}]{\mmlink{ax-mp}}
                    $\left( \φiffψ \to \φdfbiψ \right)$
                    \label{bi1:ax-mp}
                \statement{\mmurl{simplim}}
                    $\left( \φdfbiψ \to \φtoψ \right)$
                    \label{bi1:simplim:2}
                \statement[\ref{bi1:ax-mp}, \ref{bi1:simplim:2}]{\mmlink{syl}}
                    $\left( \φiffψ \to \φtoψ \right)$
            \end{mmtable}
        \end{mmproof}
    \end{theorem}
\end{frame}

\begin{frame}{雙向蘊涵化為雙條件句}
    \begin{theorem}[\mmtarget{bi3}]
        斷言:
        \[ \left( \φtoψ \to \left( \ψtoφ \to \φiffψ \right) \right) \]

        \begin{mmproof}
            \begin{mmtable}{2em}{4em}
                \statement{\mmlink{df-bi}}
                    定義 $\φiffψ$
                    \label{bi3:df-bi}
                \statement{\mmurl{simprim}}
                    $
                        \hypersetup{linkcolor=blue}
                        \left( \ref{bi3:df-bi} \to \ref{bi3:ax-mp} \right)
                    $
                    \label{bi3:simprim}
                \statement[\ref{bi3:df-bi}, \ref{bi3:simprim}]{\mmlink{ax-mp}}
                    $\left( \φdfbiψ \to \φiffψ \right)$
                    \label{bi3:ax-mp}
                \statement[\ref{bi3:ax-mp}]{\mmurl{expi}}
                    $\left( \φtoψ \to \left( \ψtoφ \to \φiffψ \right) \right)$
            \end{mmtable}
        \end{mmproof}
    \end{theorem}
\end{frame}

\begin{frame}{雙條件句的定義的簡寫}
    \begin{theorem}[\mmurl{dfbi1}]
        斷言:
        \[ \left( \φiffψ \leftrightarrow \φdfbiψ \right) \]

        \begin{mmproof}
            \begin{mmtable}{2em}{4em}
                \statement{\mmlink{df-bi}}
                    定義 $\φiffψ$
                    \label{dfbi1:df-bi}
                \statement{\mmurl{simplim}}
                    $
                        \hypersetup{linkcolor=blue}
                        \left( \ref{dfbi1:df-bi} \to \ref{dfbi1:ax-mp} \right)
                    $
                    \label{dfbi1:simplim:1}
                \statement[\ref{dfbi1:df-bi}, \ref{dfbi1:simplim:1}]{\mmlink{ax-mp}}
                    $\left( \φiffψ \to \φdfbiψ \right)$
                    \label{dfbi1:ax-mp}
                \statement{\mmlink{bi3}}
                    $\left( \φtoψ \to \left( \ψtoφ \to \φiffψ \right) \right)$
                    \label{dfbi1:bi3}
                \statement{\mmurl{impi}}
                    $\left( \φdfbiψ \to \φiffψ \right)$
                    \label{dfbi1:impi}
                \statement[\ref{dfbi1:ax-mp}, \ref{dfbi1:impi}]{\mmurl{impbii}}
                    $\left( \φiffψ \leftrightarrow \φdfbiψ \right)$
            \end{mmtable}
        \end{mmproof}
    \end{theorem}
\end{frame}

\begin{frame}{把前件換成它的定義}
    \begin{theorem}[\mmurl{sylbi}]
        假說:
        \[ \tag{sylbi.1} \left( \φ \leftrightarrow \ψ \right) \]
        \[ \tag{sylbi.2} \left( \ψ \to \χ \right) \]

        斷言:
        \[ \left( \φ \to \χ \right) \]

        \begin{mmproof}
            \begin{mmtable}{2em}{4em}
                \statement{sylbi.1}
                    $\left( \φ \leftrightarrow \ψ \right)$
                    \label{sylbi.1}
                \statement[\ref{sylbi.1}]{\mmurl{biimpi}}
                    $\left( \φ \to \ψ \right)$
                    \label{sylbi:biimpi}
                \statement{sylbi.2}
                    $\left( \ψ \to \χ \right)$
                    \label{sylbi.2}
                \statement[\ref{sylbi:biimpi}, \ref{sylbi.2}]{\mmlink{syl}}
                    $\left( \φ \to \χ \right)$
            \end{mmtable}
        \end{mmproof}
    \end{theorem}
\end{frame}

\begin{frame}{同義蘊涵反向蘊涵}
    \begin{theorem}[\mmurl{bi2}]
        斷言:
        \[ \left( \φiffψ \to \ψtoφ \right) \]

        \begin{mmproof}
            \begin{mmtable}{2em}{4em}
                \statement{\mmlink{dfbi1}}
                    $\left( \φiffψ \leftrightarrow \φdfbiψ \right)$
                    \label{bi2:dfbi1}
                \statement{\mmurl{simprim}}
                    $\left( \φdfbiψ \to \φiffψ \right)$
                    \label{bi2:simprim}
                \statement[\ref{bi2:dfbi1}, \ref{bi2:simprim}]{\mmlink{sylbi}}
                    $\left( \φiffψ \to \ψtoφ \right)$
            \end{mmtable}
        \end{mmproof}
    \end{theorem}
\end{frame}

\begin{frame}{雙條件句有交換律,單向}
    \begin{theorem}[\mmurl{bicom1}]
        斷言:
        \[ \left( \φiffψ \to \left( \ψ \leftrightarrow \φ \right) \right) \]

        \begin{mmproof}
            \begin{mmtable}{2em}{4em}
                \statement{\mmlink{bi2}}
                    $\left( \φiffψ \to \ψtoφ \right)$
                    \label{bicom1:bi2}
                \statement{\mmlink{bi1}}
                    $\left( \φiffψ \to \φtoψ \right)$
                    \label{bicom1:bi1}
                \statement[\ref{bicom1:bi2}, \ref{bicom1:bi1}]{\mmurl{impbid}}
                    $\left( \φiffψ \to \left( \ψ \leftrightarrow \φ \right) \right)$
            \end{mmtable}
        \end{mmproof}
    \end{theorem}
\end{frame}

\begin{frame}{雙條件句有交換律,雙向}
    \begin{theorem}[\mmurl{bicom}]
        斷言:
        \[ \left( \φiffψ \leftrightarrow \left( \ψ \leftrightarrow \φ \right) \right) \]

        \begin{mmproof}
            \begin{mmtable}{2em}{4em}
                \statement{\mmlink{bicom1}}
                    $\left( \φiffψ \to \left( \ψ \leftrightarrow \φ \right) \right)$
                    \label{bicom:bicom1:1}
                \statement{\mmlink{bicom1}}
                    $\left( \left( \ψ \leftrightarrow \φ \right) \to \φiffψ \right)$
                    \label{bicom:bicom1:2}
                \statement[\ref{bicom:bicom1:1}, \ref{bicom:bicom1:2}]{\mmurl{impbii}}
                    $\left( \φiffψ \leftrightarrow \left( \ψ \leftrightarrow \φ \right) \right)$
            \end{mmtable}
        \end{mmproof}
    \end{theorem}
\end{frame}
\end{document}
