% Copyright (C) 2015 Chen-Pang He <http://jdh8.org/>
%
% This file may be distributed and/or modified under
%
% 1. LaTeX Project Public License
% 2. GNU Public License
%
% See the files COPYING.* for more details.

\documentclass{Slideshow}
\usepackage{metamath}

\begin{document}
\title[邏輯]{邏輯}
\maketitle

\begin{frame}{推理}
    定理需要其他已證實的定理證明?

    \[ \wph \to \wps \to \wch \to \wth \]

    如果我們什麼都不相信,根本推理不出任何東西。

    \begin{description}
        \item[公理] 不證自明,預先設定為真。
        \item[定理] 由公理或其他已證明的定理證明為真。
    \end{description}
\end{frame}

\begin{frame}{形式系統}
    \begin{description}
        \item[文法] 用來決定符號怎麼組合成句子 (wff)\footnote{well-formed
            formula, 一般譯作「完構式」等。}
        \item[公理] 不證自明,預先設定為真的句子
        \item[推論規則] 有前提的公理
        \item[定義] 引入新符號的公理,不能使原系統證明更多句子
        \item[定理] 由公理或其他已證明的定理證明為真
    \end{description}
\end{frame}

\begin{frame}{Metamath}
    \mmhref{mmtheorems}{Metamath} 是描述形式系統的語言。文法只有簡單的取代。

    \begin{description}
        \item[句子] 用第一個字說明句子的性質
        \item[文法] 以某性質的某變數為前提,把變數組成句子
    \end{description}
\end{frame}

\section{命題邏輯}
\newcommand{\wn}{\neg}
\newcommand{\wi}[2]{\left( #1 \to #2 \right)}
\newcommand{\wb}[2]{\left( #1 \leftrightarrow #2 \right)}
\newcommand{\wa}[2]{\left( #1 \wedge #2 \right)}
\newcommand{\wo}[2]{\left( #1 \vee #2 \right)}

\begin{frame}{命題邏輯}
    命題邏輯研究句子之間的關係。

    \begin{description}
        \item[蘊涵] $\wi\wph\wps$ 是推理的核心
        \item[否定] $\wn\wph$
        \item[等價] $\wb\wph\wps$ 用來定義新符號
        \item[且] $\wa\wph\wps$
        \item[或] $\wo\wph\wps$
    \end{description}
\end{frame}

\newcommand{\aIIi}[3]{\wi{\wi{#1}{#2}}{\wi{#1}{#3}}}

\newcommand{\axI}[2]{\wi{#1}{\wi{#2}{#1}}}
\newcommand{\axII}[3]{\wi{\wi{#1}{\wi{#2}{#3}}}{\aIIi{#1}{#2}{#3}}}
\newcommand{\axIII}[2]{\wi{\wi{\wn#1}{\wn#2}}{\wi{#2}{#1}}}

\begin{frame}{\href
    {https://zh.wikipedia.org/wiki/\%E6\%89\%AC\%C2\%B7\%E6\%AD\%A6\%E5\%8D\%A1\%E8\%B0\%A2\%E7\%BB\%B4\%E5\%A5\%87}
    {Łukasiewicz} 的公理系統
}
    \[ \axI\wph\wps      \tag{\mmlink{ax-1}} \]
    \[ \axII\wph\wps\wch \tag{\mmlink{ax-2}} \]
    \[ \axIII\wph\wps    \tag{\mmlink{ax-3}} \]
    \[ \wph \ \mbox{和}\ \wi\wph\wps \ \mbox{能證明}\ \wps \tag{\mmlink{ax-mp}} \]
\end{frame}

\newcommand{\dfor}[2]{\wi{\wn#1}{#2}}
\newcommand{\dfan}[2]{\wn\wi{#1}{\wn#2}}
\newcommand{\dfbi}[2]{\dfan{\wi{#1}{#2}}{\wi{#2}{#1}}}

\begin{frame}{定義其他邏輯符號}
    $\wo\wph\wps$ 定義為 $\dfor\wph\wps$

    $\wa\wph\wps$ 定義為 $\dfan\wph\wps$

    $\wb\wph\wps$ 定義為 $\wa{\wi\wph\wps}{\wi\wps\wph}$
\end{frame}

\subsection{蘊涵}
\begin{frame}{邏輯蘊涵}
    \begin{syntax}[\mmtarget{wi}]
        假說:
        \begin{align*}
            \wff \wph \\
            \wff \wps
        \end{align*}

        斷言:
        \[ \wff \wi\wph\wps \]
    \end{syntax}
\end{frame}

\begin{frame}{公理 1:蘊涵的簡化}
    \begin{axiom}[\mmtarget{ax-1}]
        斷言:
        \[ \axI\wph\wps \]
    \end{axiom}

    如果 $\wph$ 是真理,那麼加上任意前提 $\wps$ 都成立。
\end{frame}

\begin{frame}{公理 2:蘊涵的分配律}
    \begin{axiom}[\mmtarget{ax-2}]
        斷言:
        \[ \axII\wph\wps\wch \]
    \end{axiom}

    \href
        {https://zh.wikipedia.org/wiki/\%E6\%88\%88\%E7\%89\%B9\%E6\%B4\%9B\%E5\%B8\%83\%C2\%B7\%E5\%BC\%97\%E9\%9B\%B7\%E6\%A0\%BC}
        {Frege}
    是一位想把所有的數學運算奠基在邏輯上的數學家。這條公理是他在他的公理系統中
    首先提出的。
\end{frame}

\begin{frame}{推論規則:肯定前件}
    \begin{axiom}[\mmtarget{ax-mp}]
        假說:
        \[ \wph        \tag{min} \]
        \[ \wi\wph\wps \tag{maj} \]

        斷言:
        \[ \wps \]
    \end{axiom}

    這可以用來把定理 $\wi\wph\wps$ 的 $\wps$ 兌現出來。
\end{frame}

\begin{frame}{定理、演繹、推論}
    要描述「若 $\sigma$ 則 $\tau$」有三種表示方法:
    \begin{description}
        \item[定理] 能證明 $\wi\sigma\tau$
        \item[推論] $\sigma$ 能證明 $\tau$
        \item[演繹] $\wi\wph\sigma$ 能證明 $\wi\wph\tau$
    \end{description}

    推論是最弱的形式,需要複雜的\mmhref{mmdeduction}{演繹定理}才能化為另外兩者。
    \begin{description}
        \item[定理 $\implies$ 推論] 肯定前件 (\mmlink{ax-mp})
        \item[定理 $\implies$ 演繹] 三段論 (\mmlink{syl})
        \item[演繹 $\implies$ 定理] 同一律 (\mmlink{id})
    \end{description}
\end{frame}

\begin{frame}{公理 1 的推論形式}
    \begin{theorem}[\mmtarget{a1i}]
        假說:
        \[ \wph \tag{a1i.1} \]

        斷言:
        \[ \wi\wps\wph \]

        \begin{mmtable}{2em}{3em}
            \statement{a1i.1}
                $\wph$
                \label{a1i:1}
            \statement{\mmlink{ax-1}}
                $\axI\wph\wps$
                \label{a1i:ax-1}
            \statement[\ref{a1i:1}, \ref{a1i:ax-1}]{\mmlink{ax-mp}}
                $\wi\wps\wph$
        \end{mmtable}
    \end{theorem}
\end{frame}

\begin{frame}{公理 2 的推論形式}
    \begin{theorem}[\mmtarget{a2i}]
        \newcommand{\hyp}{\wi\wph{\wi\wps\wch}}

        假說:
        \[ \hyp \tag{a2i.1} \]

        斷言:
        \[ \aIIi\wph\wps\wch \]

        \begin{mmtable}{2em}{3em}
            \statement{a2i.1}
                $\hyp$
                \label{a2i:1}
            \statement{\mmlink{ax-2}}
                $\wi\hyp{\aIIi\wph\wps\wch}$
                \label{a2i:ax-2}
            \statement[\ref{a2i:1}, \ref{a2i:ax-2}]{\mmlink{ax-mp}}
                $\aIIi\wph\wps\wch$
        \end{mmtable}
    \end{theorem}
\end{frame}

\begin{frame}{肯定前件的演繹}
    \begin{theorem}[\mmtarget{mpd}]
        假說:
        \[ \wi\wph\wps          \tag{mpd.1} \]
        \[ \wi\wph{\wi\wps\wch} \tag{mpd.2} \]

        斷言:
        \[ \wi\wph\wch \]
    \end{theorem}

    這個演繹必須額外證明,因為它要用來證明三段論。
\end{frame}

\begin{frame}{肯定前件的演繹的證明}
    \begin{mmtable}{2em}{3em}
        \statement{mpd.1}
            $\wi\wph\wps$
            \label{mpd:1}
        \statement{mpd.2}
            $\wi\wph{\wi\wps\wch}$
            \label{mpd:2}
        \statement[\ref{mpd:2}]{\mmlink{a2i}}
            $\aIIi\wph\wps\wch$
            \label{mpd:a2i}
        \statement[\ref{mpd:1}, \ref{mpd:a2i}]{\mmlink{ax-mp}}
            $\wi\wph\wch$
    \end{mmtable}
\end{frame}

\begin{frame}{三段論}
    \begin{theorem}[\mmtarget{syl}]
        假說:
        \[ \wi\wph\wps \tag{syl.1} \]
        \[ \wi\wps\wch \tag{syl.2} \]

        斷言:
        \[ \wi\wph\wch \]
    \end{theorem}
\end{frame}

\begin{frame}{三段論的證明}
    \begin{mmtable}{2em}{3em}
        \statement{syl.1}
            $\wi\wph\wps$
            \label{syl:1}
        \statement{syl.2}
            $\wi\wps\wch$
            \label{syl:2}
        \statement[\ref{syl:2}]{\mmlink{a1i}}
            $\wi\wph{\wi\wps\wch}$
            \label{syl:a1i}
        \statement[\ref{syl:1}, \ref{syl:a1i}]{\mmlink{mpd}}
            $\wi\wph\wch$
    \end{mmtable}
\end{frame}

\begin{frame}{同一律}
    \begin{theorem}[\mmtarget{id}]
        斷言:
        \[ \wi\wph\wph \]

        \begin{mmtable}{2em}{3em}
            \statement{\mmlink{ax-1}}
                $\axI\wph\wph$
                \label{id:ax-1:1}
            \statement{\mmlink{ax-1}}
                $\axI\wph{\wi\wph\wph}$
                \label{id:ax-1:2}
            \statement[\ref{id:ax-1:1}, \ref{id:ax-1:2}]{\mmlink{mpd}}
                $\wi\wph\wph$
        \end{mmtable}
    \end{theorem}
\end{frame}

\begin{frame}{公理 1 的演繹}
    \begin{theorem}[\mmtarget{a1d}]
        假說:
        \[ \wi\wph\wps \tag{a1d.1} \]

        斷言:
        \[ \wi\wph{\wi\wch\wps} \]

        \begin{mmtable}{2em}{4em}
            \statement{a1d.1}
                $\wi\wph\wps$
                \label{a1d:1}
            \statement{\mmlink{ax-1}}
                $\axI\wps\wch$
                \label{a1d:ax-1}
            \statement[\ref{a1d:1}, \ref{a1d:ax-1}]{\mmlink{syl}}
                $\wi\wph{\wi\wch\wps}$
        \end{mmtable}
    \end{theorem}
\end{frame}

\subsection{否定}
\begin{frame}{否定}
    \begin{syntax}
        假說:
        \[ \wff \wph \]

        斷言:
        \[ \wff \wn\wph \]
    \end{syntax}
\end{frame}

\begin{frame}{公理 3:換位律}
    \begin{axiom}[\mmtarget{ax-3}]
        斷言:
        \[ \axIII\wph\wps \]
    \end{axiom}

    \begin{example}
        如果天空沒有雲就不會下雨,那麼如果下雨了,天空一定有雲。
    \end{example}
\end{frame}

\begin{frame}{公理 3 的演繹}
    \begin{theorem}[\mmtarget{con4d}]
        \newcommand{\hyp}{\wi\wph{\wi{\wn\wps}{\wn\wch}}}

        假說:
        \[ \hyp \tag{con4d.1} \]

        斷言:
        \[ \wi\wph{\wi\wch\wps} \]

        \begin{mmtable}{2em}{4em}
            \statement{con4d.1}
                $\hyp$
                \label{con4d:1}
            \statement{\mmlink{ax-3}}
                $\axIII\wps\wch$
                \label{con4d:ax-3}
            \statement[\ref{con4d:1}, \ref{con4d:ax-3}]{\mmlink{syl}}
                $\wi\wph{\wi\wch\wps}$
        \end{mmtable}
    \end{theorem}
\end{frame}
\end{document}
