% Copyright (C) 2015 Chen-Pang He <http://jdh8.org/>
%
% This file may be distributed and/or modified under
%
% 1. LaTeX Project Public License
% 2. GNU Public License
%
% See the files COPYING.* for more details.

\documentclass{Slideshow}
\usepackage{metamath}

\begin{document}
\title[微積分演習課]{臺北醫學大學微積分演習課}
\maketitle

\section{前言}
\subsection{關於}
\begin{frame}{關於我}
    \begin{columns}[onlytextwidth]
        \begin{column}{0.6\textwidth}
            \begin{itemize}
                \item \href{http://my2.tmu.edu.tw/b101100025}{B101100025}
                \item \href{https://www.facebook.com/jdh863}{臉書}
                \item \href{https://plus.google.com/+\%E4\%BD\%95\%E9\%9C\%87\%E9\%82\%A6-jdh8}{Google+}
                \item 0918-319823
                    \begin{itemize}
                        \item 真是充滿火藥味的號碼
                    \end{itemize}
            \end{itemize}
        \end{column}

        \begin{column}{0.4\textwidth}
            \begin{flushleft}
                \newlength{\stickerwidth}
                \setlength{\stickerwidth}{\columnwidth - 1em}
                \includegraphics[width=\stickerwidth]{Introduction/sticker.jpg}
            \end{flushleft}
        \end{column}
    \end{columns}
\end{frame}

\begin{frame}{課程內容}
    以培養解題能力為目標。先通過考試,後培養邏輯思辨能力。

    \begin{itemize}
        \item \href{https://jdh8.github.io/calculus-slides/}{投影片}
        \item \href{http://jdh8.org/category/calculus-course/}{考古題、上課影片}
        \item 網頁教材(建構中)
            \begin{itemize}
                \item \href{http://jdh8.org/calculus/}{部落格頁面}
                \item \href{https://zh.wikibooks.org/wiki/\%E5\%BE\%AE\%E7\%A7\%AF\%E5\%88\%86\%E5\%AD\%A6}{維基教科書}
            \end{itemize}
    \end{itemize}
\end{frame}

\begin{frame}{Copyleft}
    本系列教材具有圖文創作與程式創作的性質,所以同時以下列條款釋出。

    \begin{description}
        \item[CC BY-SA] 適合圖文創作
        \item[GPL\scriptsize~v3+] 適合程式創作
    \end{description}

    Copyleft 允許閱聽人自由使用、複製、研究、修改著作,但必須以相同方式分享。

    \begin{figure}
        \href
            {https://commons.wikimedia.org/wiki/File:Copyleft.svg}
            {\includegraphics[width=5em]{Introduction/Copyleft.pdf}}
    \end{figure}
\end{frame}

\subsection{微積分是什麼}
\begin{frame}{先備知識}
    \begin{figure}
        \href
            {https://commons.wikimedia.org/wiki/File:Japanese_Senbeis.jpg}
            {\includegraphics[width=\textheight]{Introduction/Japanese_Senbeis.jpg}}
        \caption{\href{https://ja.wikipedia.org/wiki/\%E7\%85\%8E\%E9\%A4\%85}{仙貝},由 DryPot 所攝}
    \end{figure}
\end{frame}

\begin{frame}
    \begin{quote}
        If people do not believe that mathematics is simple, it is only because
        they do not realize how complicated life is.

        \begin{flushright}
            \textup{John von Neumann (1903--1957)}
        \end{flushright}
    \end{quote}
\end{frame}

\begin{frame}{微積分是什麼}
    微積分是研究\textbf{連續}、\textbf{變化}的數學。

    \begin{description}
        \item[連續] 用\textbf{實數}來表示連續
        \item[變化] 用\textbf{函數}來表示變化
    \end{description}

    微積分的主要研究對象是以實數為參數的函數。
\end{frame}

\section{邏輯}
\begin{frame}{推理}
    定理需要其他已證實的定理證明?

    \[ \wph \to \wps \to \wch \to \wth \]

    如果我們什麼都不相信,根本推理不出任何東西。

    \begin{description}
        \item[公理] 不證自明,預先設定為真。
        \item[定理] 由公理或其他已證明的定理證明為真。
    \end{description}
\end{frame}

\begin{frame}{形式系統}
    \begin{description}
        \item[文法] 用來決定符號怎麼組合成句子 (wff)\footnote{well-formed
            formula, 一般譯作「完構式」等。}
        \item[公理] 不證自明,預先設定為真的句子
        \item[推論規則] 有前提的公理
        \item[定義] 引入新符號的公理,不能使原系統證明更多句子
        \item[定理] 由公理或其他已證明的定理證明為真
    \end{description}
\end{frame}

\begin{frame}{Metamath}
    \mmhref{mmtheorems}{Metamath} 是描述形式系統的語言。文法只有簡單的取代。

    \begin{description}
        \item[句子] 用第一個字說明句子的性質
        \item[文法] 以某性質的某變數為前提,把變數組成句子
    \end{description}
\end{frame}

\subsection{命題邏輯}
\newcommand{\wn}{\neg}
\newcommand{\wi}[2]{\left( #1 \to #2 \right)}
\newcommand{\wb}[2]{\left( #1 \leftrightarrow #2 \right)}
\newcommand{\wa}[2]{\left( #1 \wedge #2 \right)}
\newcommand{\wo}[2]{\left( #1 \vee #2 \right)}

\begin{frame}{命題邏輯}
    命題邏輯研究句子之間的關係。

    \begin{description}
        \item[蘊涵] $\wi\wph\wps$ 是推理的核心
        \item[否定] $\wn\wph$
        \item[等價] $\wb\wph\wps$ 用來定義新符號
        \item[且] $\wa\wph\wps$
        \item[或] $\wo\wph\wps$
    \end{description}
\end{frame}

\newcommand{\aIIi}[3]{\wi{\wi{#1}{#2}}{\wi{#1}{#3}}}

\newcommand{\axI}[2]{\wi{#1}{\wi{#2}{#1}}}
\newcommand{\axII}[3]{\wi{\wi{#1}{\wi{#2}{#3}}}{\aIIi{#1}{#2}{#3}}}
\newcommand{\axIII}[2]{\wi{\wi{\wn#1}{\wn#2}}{\wi{#2}{#1}}}

\begin{frame}{Łukasiewicz 的公理系統}
    \[ \axI\wph\wps      \tag{\mmurl{ax-1}} \]
    \[ \axII\wph\wps\wch \tag{\mmurl{ax-2}} \]
    \[ \axIII\wph\wps    \tag{\mmurl{ax-3}} \]
    \[ \wph \quad\&\quad \wi\wph\wps \quad\implies\quad \wps \tag{\mmurl{ax-mp}} \]
\end{frame}

\newcommand{\dfor}[2]{\wi{\wn#1}{#2}}
\newcommand{\dfan}[2]{\wn\wi{#1}{\wn#2}}
\newcommand{\dfbi}[2]{\dfan{\wi{#1}{#2}}{\wi{#2}{#1}}}

\begin{frame}{定義其他邏輯符號}
    $\wo\wph\wps$ 定義為 $\dfor\wph\wps$

    $\wa\wph\wps$ 定義為 $\dfan\wph\wps$

    $\wb\wph\wps$ 定義為 $\wa{\wi\wph\wps}{\wi\wps\wph}$
\end{frame}

\subsubsection{蘊涵}
\begin{frame}{邏輯蘊涵}
    \begin{syntax}
        假說:
        \begin{align*}
            \wff \wph \\
            \wff \wps
        \end{align*}

        斷言:
        \[ \wff \wi\wph\wps \]
    \end{syntax}
\end{frame}

\begin{frame}{公理 1:蘊涵的簡化}
    \begin{axiom}[\mmurl{ax-1}]
        斷言:
        \[ \axI\wph\wps \]
    \end{axiom}

    如果 $\wph$ 是真理,那麼加上任意前提 $\wps$ 都成立。
\end{frame}

\begin{frame}{公理 2:蘊涵的分配律}
    \begin{axiom}[\mmurl{ax-2}]
        斷言:
        \[ \axII\wph\wps\wch \]
    \end{axiom}

    Frege 是一位想把所有的數學運算奠基在邏輯上的數學家。這條公理是他在他的公理
    系統中首先提出的。
\end{frame}

\begin{frame}{推論規則:肯定前件}
    \begin{axiom}[\mmurl{ax-mp}]
        假說:
        \[ \wph        \tag{min} \]
        \[ \wi\wph\wps \tag{maj} \]

        斷言:
        \[ \wps \]
    \end{axiom}

    這可以用來把定理 $\wi\wph\wps$ 的 $\wps$ 兌現出來。
\end{frame}

\begin{frame}{定理、演繹、推論}
    要描述「若 $\sigma$ 則 $\tau$」有三種表示方法:
    \begin{description}
        \item[定理] $\wi\sigma\tau$
        \item[推論] $\sigma$ 能推論出 $\tau$
        \item[演繹] $\wi\wph\sigma$ 能推論出 $\wi\wph\tau$
    \end{description}

    推論是最弱的形式,需要複雜的\mmhref{mmdeduction}{演繹定理}才能化為另外兩者。
    \begin{description}
        \item[定理 $\implies$ 推論] 肯定前件 (\mmurl{ax-mp})
        \item[定理 $\implies$ 演繹] 三段論 (\mmurl{syl})
        \item[演繹 $\implies$ 定理] 同一律 (\mmurl{id})
    \end{description}
\end{frame}

\begin{frame}{公理 1 的推論形式}
    \begin{theorem}[\mmurl{a1i}]
        假說:
        \[ \wph \tag{a1i.1} \]

        斷言:
        \[ \wi\wps\wph \]

        \begin{mmproof}{2em}{3em}
            \statement{a1i.1}
                $\wph$
                \label{a1i:1}
            \statement{\mmurl{ax-1}}
                $\axI\wph\wps$
                \label{a1i:2}
            \statement[\ref{a1i:1}, \ref{a1i:2}]{\mmurl{ax-mp}}
                $\wi\wps\wph$
        \end{mmproof}
    \end{theorem}
\end{frame}

\begin{frame}{公理 2 的推論形式}
    \begin{theorem}[\mmurl{a2i}]
        \newcommand{\hyp}{\wi\wph{\wi\wps\wch}}

        假說:
        \[ \hyp \tag{a2i.1} \]

        斷言:
        \[ \aIIi\wph\wps\wch \]

        \begin{mmproof}{2em}{3em}
            \statement{a2i.1}
                $\hyp$
                \label{a2i:1}
            \statement{\mmurl{ax-2}}
                $\wi\hyp{\aIIi\wph\wps\wch}$
                \label{a2i:2}
            \statement[\ref{a2i:1}, \ref{a2i:2}]{\mmurl{ax-mp}}
                $\aIIi\wph\wps\wch$
        \end{mmproof}
    \end{theorem}
\end{frame}

\begin{frame}{肯定前件的演繹}
    \begin{theorem}[\mmurl{mpd}]
        假說:
        \[ \wi\wph\wps          \tag{mpd.1} \]
        \[ \wi\wph{\wi\wps\wch} \tag{mpd.2} \]

        斷言:
        \[ \wi\wph\wch \]
    \end{theorem}

    這個演繹必須額外證明,因為它要用來證明三段論。
\end{frame}

\begin{frame}{肯定前件的演繹的證明}
    \begin{mmproof}{2em}{3em}
        \statement{mpd.1}
            $\wi\wph\wps$
            \label{mpd:1}
        \statement{mpd.2}
            $\wi\wph{\wi\wps\wch}$
            \label{mpd:2}
        \statement[\ref{mpd:2}]{\mmurl{a2i}}
            $\aIIi\wph\wps\wch$
            \label{mpd:3}
        \statement[\ref{mpd:1}, \ref{mpd:3}]{\mmurl{ax-mp}}
            $\wi\wph\wch$
    \end{mmproof}
\end{frame}

\begin{frame}{三段論}
    \begin{theorem}[\mmurl{syl}]
        假說:
        \[ \wi\wph\wps \tag{syl.1} \]
        \[ \wi\wps\wch \tag{syl.2} \]

        斷言:
        \[ \wi\wph\wch \]
    \end{theorem}
\end{frame}

\begin{frame}{三段論的證明}
    \begin{mmproof}{2em}{3em}
        \statement{syl.1}
            $\wi\wph\wps$
            \label{syl:1}
        \statement{syl.2}
            $\wi\wps\wch$
            \label{syl:2}
        \statement[\ref{syl:2}]{\mmurl{a1i}}
            $\wi\wph{\wi\wps\wch}$
            \label{syl:3}
        \statement[\ref{syl:1}, \ref{syl:3}]{\mmurl{mpd}}
            $\wi\wph\wch$
    \end{mmproof}
\end{frame}

\begin{frame}{同一律}
    \begin{theorem}[\mmurl{id}]
        斷言:
        \[ \wi\wph\wph \]

        \begin{mmproof}{2em}{3em}
            \statement{\mmurl{ax-1}}
                $\axI\wph\wph$
                \label{id:1}
            \statement{\mmurl{ax-1}}
                $\axI\wph{\wi\wph\wph}$
                \label{id:2}
            \statement[\ref{id:1}, \ref{id:2}]{\mmurl{mpd}}
                $\wi\wph\wph$
        \end{mmproof}
    \end{theorem}
\end{frame}

\section{集合}
\begin{frame}{集合}
    數學是研究\textbf{數}和\textbf{形}的學問。數學家發現他們都可以用\textbf{集合}來表示。

    數學的宇宙就是\href
        {https://zh.wikipedia.org/wiki/\%E5\%86\%AF\%C2\%B7\%E8\%AF\%BA\%E4\%BC\%8A\%E6\%9B\%BC\%E5\%85\%A8\%E9\%9B\%86}
        {集合的宇宙}。

    \begin{itemize}
        \item 集合可以等於其他集合
        \item 集合內只有集合
        \item 集合一定屬於另一個集合
    \end{itemize}

    集合可以透過公理化集合論定義。
\end{frame}

\begin{frame}{無序對}
    \begin{definition}
        \newcommand{\thepair}{\left\{ A, B \right\}}
        \mmhref{ax-pr}{配對公理}讓我們可以從兩集合 $A$, $B$ 建構出新的集合
        \[ \thepair \]
        使得 $x \in \thepair$ 就是指 $x = A$ 或 $x = B$。
    \end{definition}

    如果 $A = B$ 就變成了 $\left\{ A, A \right\}$ 可以進一步簡寫為
    \[ \left\{ A \right\}.\]
\end{frame}

\begin{frame}{序對}
    \begin{definition}
        為了確保 $\left\langle A, B \right\rangle = \left\langle C, D \right\rangle$ 意同
        \[ A = C \quad \mbox{且} \quad B = D,\]

        可以定義
        \begin{equation}
            \left\langle A, B \right\rangle =
            \left\{ \left\{ A \right\}, \left\{ A, B \right\} \right\}.
            \label{eq:df-op}
        \end{equation}
    \end{definition}

    \eqref{eq:df-op} 不是從集合定義序對的唯一方法。
\end{frame}

\subsection{關係與函數}
\begin{frame}{關係}
    \begin{definition}
        一般我們把 $A$, $B$, $R$ 三個集合寫在一起的句子
        \[ ARB \]

        視為 $\left\langle A, B \right\rangle \in R$ 的簡寫。
    \end{definition}

    \begin{example}
        \[ 1 < 2 \quad \mbox{即} \quad \left\langle 1, 2 \right\rangle \in\ < .\]
    \end{example}
\end{frame}

\begin{frame}{反關係}
    \begin{definition}
        反關係就是把關係的左右參數互換。
        \[ A R^{-1} B \quad \mbox{即} \quad BRA.\]
    \end{definition}

    \begin{example}
        \[ > \ = \ <^{-1}.\]
    \end{example}
\end{frame}

\begin{frame}{複合關係}
    \begin{definition}
        滿足複合關係,意味著存在著「中間人」串起兩個關係。

        \[ A \left( F \circ G \right) B \]

        即存在一個 $x$ 使得
        \[ AGx \quad \mbox{且} \quad xFB \]
    \end{definition}

    \begin{example}
        \[ \mbox{姑姑} = \left( \mbox{姐妹} \circ \mbox{爸爸} \right). \]
    \end{example}
\end{frame}
\end{document}
